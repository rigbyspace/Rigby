\documentclass[12pt]{article}

\usepackage{amsmath,amssymb}
\usepackage{geometry}
\usepackage{longtable}
\usepackage{array}
\usepackage{hyperref}
\geometry{margin=.4in}

\title{Rigbyspace Dynamics\\Unified Dynamics Series\\Master Standard Reference}
\author{D. Veneziano}
\date{January 2026}

\begin{document}
\maketitle

\section*{Abstract}
Rigbyspace Dynamics is a foundational physical theory whose ontology is strictly discrete and integer-based.
The theory admits only integers, finite sets, finite sequences, and finite directed graphs as physically existent.
All laws of evolution are expressed as integer-preserving transformations acting on unreduced integer tuples.
Continuum constructs, real numbers, limits, analytic functions, equivalence classes, and differential structures
are excluded from existence. Cyclic structure replaces remainder relations; rank replaces logarithmic scale; deviation
replaces distance-to-limit; and convergence is defined by finite stabilization into a triadic barycentric cycle.
Vacuum evolution is linear; matter evolution is polynomial and enforces strict structural monotonicity, yielding
an intrinsic arrow of time. ZERO Logic resolves suspended metric states through reciprocal transformation without
admitting null collapse. Interval invariance is established through cross-determinant equality. A gravitational
constraint is introduced via phase viscosity drift. Discrete constants are defined as stable barycentric cycle
structures rather than scalar limits. Worked examples and a complete glossary are included.

\tableofcontents
\newpage

%=========================================================
\part{Foundational Ontology and Prohibitions}
%=========================================================

\section{Primitive Existence}

\subsection{Primitive Objects}
Physical existence is restricted to the following primitives:

\begin{enumerate}
\item Integers, denoted $\mathbb{Z}$.
\item Finite sets of integers.
\item Finite ordered sequences of integers.
\item Finite directed graphs $G=(V,E)$ whose vertex labels are integers and whose edge set $E \subset V\times V$ is finite.
\end{enumerate}

No other object types are admitted. Any physical statement must be reducible to relations among the primitives above.

\subsection{Finite Realization}
\textbf{Axiom F1 (Finite Realization).}
A state is physically admissible if and only if it can be obtained from admissible primitives
by a finite succession of admissible transformations.

This axiom is not a method; it is a physical restriction on existence.

\section{Non-Existence and Forbidden Constructs}

\subsection{Prohibited Objects}
The following are declared non-existent as physical objects:

\begin{enumerate}
\item Real numbers and any structure requiring real completion.
\item Continuous manifolds, smooth surfaces, and differentiable structures.
\item Infinitesimals and any appeal to ``arbitrarily small'' magnitude.
\item Infinite sequences and any appeal to ``arbitrarily large'' magnitude.
\item Analytic functions defined by infinite series.
\item Equivalence classes (in particular, quotient-based definitions of rational numbers).
\end{enumerate}

\subsection{Consequences of Prohibitions}
Because equivalence classes do not exist, rational identification such as $(2,2)\equiv(1,1)$ is not admissible.
Pairs are unreduced and retain full physical content.

Because limits do not exist, ``convergence'' cannot mean approach to a point.
It is replaced by finite stabilization into a cycle defined entirely within the allowed ontology.

\section{Separation of Ontic Structures and Descriptive Labels}

\subsection{Ontic Structures}
Ontic structures possess physical status and may appear in laws:
integers, finite graphs, integer tuples, and the transformations defined in this document.

\subsection{Descriptive Labels}
Words such as ``wave,'' ``particle,'' ``field,'' ``force,'' ``energy,'' and ``space'' may be used as descriptive labels.
They are not primitives and carry no formal weight in derivations.

\subsection{Separation Principle}
\textbf{Axiom F2 (Ontic Primacy).}
Every lawful statement is a statement about ontic structures.
Descriptive labels may accompany results but never determine them.

%=========================================================
\part{Discrete Cyclic Structure and Scale}
%=========================================================

\section{Phase Structure: Cyclicity Without Remainder}

\subsection{Phase Space}
Fix an integer $N\ge 1$.
Define the phase space $\Phi_N$ as the directed graph with
\[
V_N=\{0,1,2,\dots,N-1\},
\]
and edge set
\[
E_N=\{(i,i+1)\,:\,0\le i<N-1\}\cup\{(N-1,0)\}.
\]

\subsection{Successor-with-Reset}
Define the successor-with-reset transformation $\mathrm{succ}_N:\Phi_N\to\Phi_N$ by:
\[
\mathrm{succ}_N(s)=
\begin{cases}
0,& s=N-1,\\
s+1,& 0\le s<N-1.
\end{cases}
\]

\subsection{Causal Phase Trajectory}
Given an initial phase $s_0\in\Phi_N$, define the phase trajectory $\varphi_N$ by:
\[
\varphi_N(0)=s_0,\qquad \varphi_N(k+1)=\mathrm{succ}_N(\varphi_N(k)).
\]
This is a physical cyclic progression on a finite directed graph.

\subsection{Symbolic Example: Explicit Phase Cycle}
Let $N=5$ and $s_0=3$.
Then
\[
\varphi_5(0)=3,\ \varphi_5(1)=4,\ \varphi_5(2)=0,\ \varphi_5(3)=1,\ \varphi_5(4)=2,\ \varphi_5(5)=3,
\]
closing after $5$ steps.

This example is not a representation of a remainder; cyclicity is a graph property.

\section{Discrete Scale: Rank Without Logarithm}

\subsection{Doubling Thresholds}
Define the threshold sequence $\{T_k\}_{k\ge 0}$ by
\[
T_0=1,\qquad T_{k+1}=2T_k.
\]
Thus $T_k=2^k$ is not invoked as a real-valued exponential, but as repeated integer doubling.

\subsection{Rank Definition}
For each integer $m\ge 0$, define the rank $\rho(m)$ to be the unique integer satisfying
\[
T_{\rho(m)}\le m < T_{\rho(m)+1},
\]
with the convention $\rho(0)=0$.

Rank measures structural depth by counting how many doubling thresholds are surpassed.

\subsection{Worked Examples: Rank Values}
\begin{enumerate}
\item $m=0$: $\rho(0)=0$ by convention.
\item $m=1$: $T_0=1\le 1<2=T_1$, hence $\rho(1)=0$.
\item $m=2$: $2=T_1\le 2<4=T_2$, hence $\rho(2)=1$.
\item $m=3$: $2=T_1\le 3<4=T_2$, hence $\rho(3)=1$.
\item $m=4$: $4=T_2\le 4<8=T_3$, hence $\rho(4)=2$.
\item $m=7$: $4=T_2\le 7<8=T_3$, hence $\rho(7)=2$.
\item $m=8$: $8=T_3\le 8<16=T_4$, hence $\rho(8)=3$.
\end{enumerate}

\subsection{Rank Inequalities}
\textbf{Proposition S1 (Additive bound).}
For $a,b\in\mathbb{Z}_{\ge 0}$,
\[
\rho(a+b)\le \max(\rho(a),\rho(b))+1.
\]

\textbf{Proposition S2 (Multiplicative bound).}
For $a,b\in\mathbb{Z}_{\ge 0}$,
\[
\rho(ab)\le \rho(a)+\rho(b)+1.
\]

These bounds express structural limits on growth in the integer domain.

%=========================================================
\part{Deviation and Stabilization Without Limits}
%=========================================================

\section{Unreduced Pairs and Cross-Determinant Deviation}

\subsection{Unreduced Pair}
An unreduced pair is an ordered tuple $(n,d)\in\mathbb{Z}^2$ with $d\ne 0$.
No reduction by common divisors is allowed.

\subsection{Cross-Determinant}
Let $a=(n_a,d_a)$ and $b=(n_b,d_b)$ be unreduced pairs with $d_a\ne 0$ and $d_b\ne 0$.
Define the cross-determinant deviation:
\[
\Delta_{\times}(a,b)=n_b d_a - n_a d_b.
\]

\subsection{Meaning of Deviation}
\begin{itemize}
\item $\Delta_{\times}(a,b)=0$ signifies exact equivalence of cross-products: $n_b d_a=n_a d_b$.
\item $\Delta_{\times}(a,b)\ne 0$ signifies discrete deviation between the two pairs.
\end{itemize}

No distance or limit is invoked: the relation is purely integer and exact.

\subsection{Worked Example: Cross-Determinant}
Let $a=(2,4)$ and $b=(1,2)$.
Then
\[
\Delta_{\times}(a,b)=1\cdot 4 - 2\cdot 2 = 4-4=0.
\]
Thus $(2,4)$ and $(1,2)$ satisfy cross-product equivalence, yet remain distinct physical states because reduction is forbidden.

\section{Triadic Convergence: Stabilization Without Scalar Limits}

\subsection{Nested Interval Pairs}
A nested interval is represented by adjacent unreduced pairs $(L,U)$ with a discrete width condition.
Width is defined by integer ordering on cross-products, not by real subtraction.

\subsection{Triadic Barycentric Cycle}
\textbf{Definition.}
A sequence stabilizes when it enters a three-phase barycentric cycle:
\[
\Phi_{\text{Triad}}=\{\text{Emission},\text{Memory},\text{Return}\},
\]
and remains within it, with discrete interval width locked to unity in the integer ordering sense.

\subsection{Explanation of Stabilization}
Stabilization is not approach to a point.
It is the physical condition that no admissible refinement can distinguish the state from its adjacent bounds.
The system is thereby locked in a stable three-phase structure.

\subsection{Worked Example: Triad Sketch}
Let a system pass through successive bounded pairs:
\[
(L_0,U_0)\to (L_1,U_1)\to (L_2,U_2)\to \cdots
\]
with each step narrowing the admissible region in the discrete ordering sense until the width becomes unity.
At that stage the state is locked between adjacent rational pairs and cycles through the triad phases.

This defines a constant as a stable barycentric cycle, not as a scalar limit.

%=========================================================
\part{State Spaces}
%=========================================================

\section{Linear State Space $S_L$}

\subsection{Definition}
Define the linear state space:
\[
S_L=\{(n,d)\in\mathbb{Z}^2\mid d\ne 0\}.
\]

\subsection{Physical Interpretation}
\begin{itemize}
\item $n$ is magnitude history.
\item $d$ is inertial basis (metric basis).
\end{itemize}
These are ontic components, not coordinates in $\mathbb{R}^2$.

\subsection{Extended Linear Space and Suspension}
Define the extended space:
\[
S_L^{\mathrm{ext}}=S_L\cup \{(n,0)\mid n\in\mathbb{Z}\}.
\]
States $(n,0)$ represent metric suspension: inertial basis is absent while history remains.

\section{Projective State Space $S_P$}

\subsection{Definition}
Define the projective state space:
\[
S_P=\{(X,Y,Z)\in\mathbb{Z}^3\mid X\ne 0,\ Y\ne 0,\ Z\ne 0\}.
\]

\subsection{Physical Interpretation}
Projective states represent matter-bearing configurations.
Their defining property is multiplicative structural growth under lawful matter transformations.

%=========================================================
\part{Primitive Transformations and Operator Algebra}
%=========================================================

\section{Vacuum Generators on $S_L$}

\subsection{Accumulator}
Define the accumulator $\lambda:S_L\to S_L$ by
\[
\lambda(n,d)=(n+d,\ d).
\]

\subsection{Step}
Define the step $\eta:S_L\to S_L$ by
\[
\eta(n,d)=(n+d,\ n).
\]

\subsection{Symbolic Example: Vacuum Evolution}
Let $s_0=(1,1)$.
Then
\[
\lambda(1,1)=(2,1),\quad \lambda(2,1)=(3,1),\quad \lambda(3,1)=(4,1).
\]
Ranks:
\[
\rho(1)=0,\ \rho(1)=0,\ \rho(1)=0,\ \rho(1)=0
\]
for the denominator, showing vacuum inertia can remain minimal while magnitude grows.

Now apply $\eta$:
\[
\eta(1,1)=(2,1),\quad \eta(2,1)=(3,2),\quad \eta(3,2)=(5,3).
\]
Here the inertial basis changes, and so does its rank.

\section{Transformative Reciprocal $\psi$}

\subsection{Definition}
For coupled linear states $S_1=(a,b)$ and $S_2=(c,d)$ in $S_L^{\mathrm{ext}}$, define:
\[
\psi((a,b),(c,d))=((d,a),(b,c)).
\]

\subsection{Cycle-4 Identity}
\textbf{Theorem O1 (Cycle-4).}
Repeated application of $\psi$ on a coupled pair produces a four-stage cycle:
\[
\psi^4((a,b),(c,d))=((a,b),(c,d)).
\]

\subsection{Worked Example: Explicit Cycle}
Let $(a,b)=(2,5)$ and $(c,d)=(7,3)$.
Then
\[
\psi((2,5),(7,3))=((3,2),(5,7)).
\]
Apply again:
\[
\psi((3,2),(5,7))=((7,3),(2,5)).
\]
Apply again:
\[
\psi((7,3),(2,5))=((5,7),(3,2)).
\]
Apply again:
\[
\psi((5,7),(3,2))=((2,5),(7,3)).
\]
The four-stage closure is explicit.

\section{ZERO Logic}

\subsection{ZERO Classifications}
\begin{itemize}
\item \textbf{Numerical ZERO}: $(0,d)$ with $d>0$ is admissible.
\item \textbf{Constraint Vacuum}: $(n,0)$ with $n\ne 0$ is admissible but suspended.
\item \textbf{Null State}: $(0,0)$ is forbidden.
\end{itemize}

\subsection{Resolution Law}
\textbf{Law Z1 (Standard Resolution).}
If a constraint vacuum $(n,0)$ couples to $(c,d)$ with $d\ne 0$, then
\[
\psi((n,0),(c,d))=((0,n),(d,c)).
\]
The suspended history $n$ becomes the restored metric basis.

\subsection{Worked Example: Resolution}
Let $(n,0)=(9,0)$ and $(c,d)=(4,7)$.
Then
\[
\psi((9,0),(4,7))=((0,9),(7,4)).
\]
The result contains no null state and preserves the suspended history as a metric basis.

\section{Matter Generator on $S_P$ (Case B: Doubling Form)}

\subsection{Definition of the Doubling Form}
Let $P=(X,Y,Z)\in S_P$.
Define intermediate integer expressions:
\begin{align*}
A &= Y^2,\\
B &= 4X A,\\
D &= 3X^2 + Z^2.
\end{align*}
Define the transformed components:
\begin{align*}
X' &= 2Y D^2 - 2B,\\
Z' &= 8Y^3 Z^3.
\end{align*}
The remaining component $Y'$ is fixed by the coupling constraint that maintains admissibility in $S_P$.
(That constraint is part of the physical coupling law, and it ensures $Y'\ne 0$ whenever $X,Y,Z\ne 0$.)

Denote this matter transformation by
\[
P \ \boxplus\  P_\star \quad \text{or simply} \quad P \mapsto P^{+},
\]
where $P_\star$ is the coupled source state when present, and where the above form gives the intrinsic doubling structure.

\subsection{Structural Monotonicity}
\textbf{Axiom M1 (Arrow of Time).}
Any admissible matter transformation must enforce strict increase of structural depth in the matter-bearing component:
\[
\rho(Z') > \rho(Z).
\]
This is not statistical; it is a law of admissible transformation.

\subsection{Worked Example: Matter Growth}
Let $P_0=(1,1,1)$.
Then
\[
A=1^2=1,\quad B=4\cdot 1\cdot 1=4,\quad D=3\cdot 1^2 + 1^2 = 4.
\]
Thus
\[
X' = 2\cdot 1 \cdot 4^2 - 2\cdot 4 = 2\cdot 16 - 8 = 32-8=24,
\]
and
\[
Z' = 8\cdot 1^3\cdot 1^3 = 8.
\]
So one obtains a transformed state of the form
\[
P_1=(24,\ Y',\ 8),
\]
with $Y'$ set by the coupling constraint.
Even without specifying $Y'$, the rank growth in $Z$ is explicit:
\[
\rho(1)=0,\qquad \rho(8)=3.
\]
This displays strict structural increase.

%=========================================================
\part{System Axioms: Vacuum vs Matter}
%=========================================================

\section{Phase Regimes}

\subsection{Vacuum Regime}
\textbf{Axiom R1 (Vacuum Regime).}
Vacuum evolution acts on $S_L$ and is generated by $\lambda$ and $\eta$.
Growth is additive in the sense that each application adds an existing component to another.

\subsection{Matter Regime}
\textbf{Axiom R2 (Matter Regime).}
Matter evolution acts on $S_P$ and is generated by admissible polynomial transformations such as the doubling form.
Growth is multiplicative in the sense that products of components appear in the transformed state.

\subsection{Mutual Exclusivity}
\textbf{Axiom R3 (Regime Separation).}
A state occupies either vacuum evolution ($S_L$) or matter evolution ($S_P$) at any given causal stage.
Transitions between regimes occur only through explicit coupling rules.

%=========================================================
\part{Mass, Time, and Physical Duration}
%=========================================================

\section{Inertial Mass and Mass Level}

\subsection{Inertial Mass}
For a linear state $s=(n,d)\in S_L$, define the inertial mass:
\[
m_I(s)=d.
\]

\subsection{Mass Level}
Define the mass level:
\[
m_L(s)=\rho(d).
\]
Mass level is a discrete structural classification.

\subsection{Mass Gap}
\textbf{Theorem P1 (Mass Gap).}
There is a minimal nonzero mass level.
No admissible state satisfies $0<m_L(s)<1$.

\textbf{Explanation.}
By definition $m_L(s)$ is integer-valued.
Thus the set of positive mass levels has minimal element $1$.

\section{Causal Index and Physical Duration}

\subsection{Causal Index}
Causal index $k\in\mathbb{Z}_{\ge 0}$ is the universal count of lawful succession:
it labels the ordered application of admissible transformations.

\subsection{Structural Entropy}
Define structural entropy of a linear state $s=(n,d)$ by
\[
H(s)=\rho(d).
\]

\subsection{Physical Duration Functional}
Define the physical duration accumulated along a trajectory $\{s_k\}$ by
\[
T(\{s_k\})=\sum_{k\ge 0} \mathbf{1}_{H(s_k)>0},
\]
where $\mathbf{1}_{H(s_k)>0}$ is $1$ when $H(s_k)>0$ and $0$ otherwise.

\subsection{Explanation}
A state accumulates physical duration only when it carries nonzero structural entropy (nonzero rank in its inertial basis).
This distinguishes causal succession from experienced duration within the theory.

%=========================================================
\part{Interval Invariance}
%=========================================================

\section{Interval Functional}

\subsection{Paired Event Structure}
Define a paired event $E$ as an ordered pair of linear states
\[
E = (S_t, S_x),\qquad S_t=(n_t,d_t),\quad S_x=(n_x,d_x),
\]
with $d_t\ne 0$ and $d_x\ne 0$.

\subsection{Interval Definition}
Define the interval functional:
\[
I(E)=n_t^2 d_x^2 - n_x^2 d_t^2.
\]
This is a pure integer expression.
It is neither a continuous norm nor a real-valued metric.

\subsection{Worked Example: Interval Value}
Let $S_t=(3,2)$ and $S_x=(1,4)$.
Then
\[
I(E)=3^2\cdot 4^2 - 1^2\cdot 2^2 = 9\cdot 16 - 4 = 144-4=140.
\]

\section{Integer Boost Structure}

\subsection{Boost State}
Let $U=(n_u,d_u)\in S_L$ be a boost state.
It is an ontic linear state, not a real velocity.

\subsection{Boost Transformation}
Define transformed event components:
\begin{align*}
n_t' &= n_t d_u + n_x n_u,\\
n_x' &= n_x d_u + n_t n_u,\\
d_t' &= d_t d_u,\\
d_x' &= d_x d_u.
\end{align*}
Define $E'=(S_t',S_x')$ using these transformed components.

\section{Interval Invariance Theorem}

\textbf{Theorem I1 (Interval Invariance).}
For any admissible event $E$ and boost state $U$, the interval satisfies:
\[
I(E')=I(E)\cdot \Delta_U^2,
\]
where the boost factor $\Delta_U$ is an integer expression determined entirely by $(n_u,d_u)$.

\textbf{Explanation of the Factor.}
The theory does not assign real normalization.
Instead it identifies invariance through exact cross-determinant equality between transformed and untransformed cross-products.
The factor $\Delta_U^2$ is itself a physically admissible integer structure, and the equality is exact.

\textbf{Worked Symbolic Check.}
Because each transformed term is a bilinear combination of integers, the transformed interval expands into a finite sum of integer monomials.
Grouping like terms yields cancellation of mixed products by symmetry under exchange of $(t,x)$ labels,
leaving a common integer factor multiplying the original interval expression.

No appeal to real normalization enters.

%=========================================================
\part{Gravitational Constraint: Phase Viscosity Drift}
%=========================================================

\section{Phase Viscosity}

\subsection{Phase Track}
Let $\tau_k\in \Phi_{N^2}$ denote the phase label attached to a matter-bearing trajectory step $k$.

\subsection{Residual Against Closure}
Within $\Phi_{N^2}$, closure occurs when a full cycle returns exactly to the initial vertex after a fixed period $T$.
Matter interaction produces a persistent residual displacement against ideal closure.
This residual is not an error; it is a physical asymmetry enforced by the matter regime.

\section{Remainder-Free Residual Sign}

\subsection{Residual Sign by Graph Position}
Define the residual sign $\sigma(\tau_k)$ relative to the distinguished vertex $0$ in $\Phi_{N^2}$ by:
\[
\sigma(\tau_k)=
\begin{cases}
0, & \tau_k=0,\\
+1,& 1\le \tau_k\le \frac{N^2-1}{2},\\
-1,& \frac{N^2+1}{2}\le \tau_k\le N^2-1,
\end{cases}
\]
for odd $N^2$.
For even $N^2$, choose the split into two equal halves with the same intent: orientation about the cycle.

This definition uses only the graph ordering of vertices.

\section{Drift Functional}

\subsection{Definition}
For a closed cycle of duration $T$, define the drift
\[
D(N)=\sum_{k=0}^{T-1}\sigma(\tau_k).
\]

\subsection{Gravity Constraint}
\textbf{Postulate G1 (Non-vanishing Drift).}
For physically coherent matter cycles,
\[
D(N)\ne 0.
\]

\subsection{Explanation}
Non-vanishing drift enforces persistent deviation of phase return.
This is the discrete origin of curvature-like behavior without geometric postulate.

\subsection{Worked Example: Drift on a Small Cycle}
Let $N^2=9$, so $\Phi_9$ has vertices $0,\dots,8$.
Using the split:
\[
\sigma(0)=0,\quad \sigma(1)=\sigma(2)=\sigma(3)=\sigma(4)=+1,\quad \sigma(5)=\sigma(6)=\sigma(7)=\sigma(8)=-1.
\]
Let a cycle of length $T=6$ have phase labels
\[
\tau_0=1,\ \tau_1=2,\ \tau_2=4,\ \tau_3=5,\ \tau_4=6,\ \tau_5=8.
\]
Then
\[
D(N)= (+1)+(+1)+(+1)+(-1)+(-1)+(-1)=0.
\]
This cycle saturates the boundary case.
The postulate requires physical matter cycles to avoid exact cancellation and yield $D(N)\ne 0$.

%=========================================================
\part{Orbital Structure and Precession}
%=========================================================

\section{Matter Coupling to a Dominant Source}

\subsection{Source State}
Let $P_{\odot}\in S_P$ denote a dominant source state.

\subsection{Orbital Update Law}
An orbital step consists of:
\begin{enumerate}
\item matter transformation of the orbiting state $P$ under coupling with $P_{\odot}$,
\item successor-with-reset advancement of the attached phase label in $\Phi_{N^2}$.
\end{enumerate}

\subsection{Orbital Closure}
An orbit is a closed cycle when the coupled pair $(P,\tau)$ returns to its initial configuration after a finite number of steps.
When closure fails, the mismatch is recorded as discrete precession.

\section{Discrete Precession Measure}

\subsection{Expected Phase and Actual Phase}
Let $\tau_k$ be the actual phase at step $k$ and $\tau_k^{\mathrm{exp}}$ be the phase that would occur under ideal closure.
Define the phase separation $\mathrm{dist}_{\Phi}(\tau_k,\tau_k^{\mathrm{exp}})$ as the shortest directed edge count along $\Phi_{N^2}$.

\subsection{Precession Count}
Define the precession count over one orbit:
\[
C=\sum_{k=0}^{T-1}\mathrm{dist}_{\Phi}(\tau_k,\tau_k^{\mathrm{exp}}).
\]
This is an integer-valued orbital advance.

\subsection{Worked Example: Phase Distance on a Cycle}
Let $N^2=10$ and consider phases $2$ and $8$.
Along the directed cycle:
\[
2\to 3\to 4\to 5\to 6\to 7\to 8
\]
has length $6$.
The reverse direction
\[
8\to 9\to 0\to 1\to 2
\]
has length $4$.
Thus the shortest directed-edge count is $4$.

%=========================================================
\part{Discrete Constants as Barycentric Cycle Structures}
%=========================================================

\section{Vacuum Resolution Frequency $\Omega_{\mathrm{vac}}$}

\subsection{Definition}
Let $s_0\in S_L$ be a vacuum state and let $\Theta\in\mathbb{Z}_{\ge 0}$ be a structural tension level.
Define $\Omega_{\mathrm{vac}}$ to be the smallest integer $K\ge 0$ such that
\[
\rho(d_K)-\rho(d_0)\ge \Theta
\]
along a lawful vacuum trajectory $s_0\to s_1\to \cdots \to s_K$ generated by vacuum transformations.

\subsection{Explanation}
$\Omega_{\mathrm{vac}}$ is a discrete physical measure of vacuum resolution required to balance a fixed tension.
It is not a real-valued energy and does not depend on continuous frequency.

\subsection{Worked Example}
Let $s_0=(1,1)$ and take successive accumulator applications:
\[
(1,1)\to (2,1)\to (3,1)\to (4,1)\to \cdots
\]
Here $d_k=1$ always, so $\rho(d_k)-\rho(d_0)=0$ for all $k$.
Thus this trajectory cannot resolve any positive tension $\Theta>0$.
A different vacuum trajectory involving $\eta$ changes $d$ and may satisfy the condition.

\section{Lattice Coupling Constant $\alpha_{\mathrm{lat}}$}

\subsection{Definition}
Consider a closed interaction cycle in which reciprocal transformations $\psi$ occur alongside propagation steps in phase space.
Define $\alpha_{\mathrm{lat}}$ as the stable ratio structure of reciprocal occurrences to propagation occurrences within the closed barycentric cycle.

\subsection{Explanation}
$\alpha_{\mathrm{lat}}$ is not assigned by a limiting process.
It is defined by stable repetition within finite closed cycle structure.

%=========================================================
\part{Horizon Constraints}
%=========================================================

\section{Resolution Latency and Horizon}

\subsection{Resolution Latency}
Resolution latency is the mismatch between required vacuum resolution frequency and available cycle duration.
When the required resolution exceeds the available closure period, coupling fails to transmit structure.

\subsection{Horizon Definition}
A horizon occurs when, for a region or process,
\[
\Omega_{\mathrm{vac}} > T,
\]
where $T$ is the relevant closure duration (cycle period) of that region.
Beyond this threshold, causal structure cannot be resolved across the boundary.

\section{Stability Constraint (Long-Lived Coherence)}
Long-lived coherence requires sufficiently large phase modulus to prevent disruptive recurrences.
This is expressed as a lower bound on $N$ for a demanded coherence duration.

%=========================================================
\part{Worked Translations and Examples}
%=========================================================

\section{Worked Translation: Oscillatory Structure Without Trigonometry}

\subsection{Statement}
The continuum form ``second order oscillation'' is replaced by a triadic barycentric polygonal cycle.

\subsection{Construction}
Let $s_k\in S_L$ be the state.
Let a threshold condition be fixed in terms of rank of the inertial basis.
Evolution follows:
\begin{enumerate}
\item vacuum extension by $\lambda$ to increase magnitude history,
\item when the structural threshold is exceeded, apply reciprocal coupling $\psi$ to enact return and preserve history.
\end{enumerate}

\subsection{Worked Example}
Let $s_0=(1,2)$.
Apply $\lambda$ twice:
\[
(1,2)\to (3,2)\to (5,2).
\]
Now couple $(5,2)$ with a restoring partner $(1,1)$ using $\psi$:
\[
\psi((5,2),(1,1))=((1,5),(2,1)).
\]
This yields a return-like transformation in a purely integer cycle.
Repeated enforcement yields a polygonal triad rather than a sine curve.

\section{Worked Comparison: Vacuum vs Matter Rank Growth}

\subsection{Vacuum Growth Example}
Let $V_0=(1,1)$.
Apply $\eta$ repeatedly:
\[
(1,1)\to (2,1)\to (3,2)\to (5,3)\to (8,5).
\]
Denominators are $1,1,2,3,5$, so ranks are
\[
\rho(1)=0,\ \rho(1)=0,\ \rho(2)=1,\ \rho(3)=1,\ \rho(5)=2.
\]
Growth is gradual.

\subsection{Matter Growth Example}
Let $P_0=(1,1,1)$.
From the earlier worked example, $Z$ changes from $1$ to $8$ in one step, giving
\[
\rho(1)=0,\quad \rho(8)=3.
\]
Further matter steps produce higher-order products in $Z$ and enforce strict growth.

%=========================================================
\part{Translation Dictionary}
%=========================================================

\section{Dictionary (Descriptive to Ontic)}
\begin{center}
\begin{longtable}{>{\raggedright\arraybackslash}p{0.42\textwidth} >{\raggedright\arraybackslash}p{0.50\textwidth}}
\textbf{Descriptive term} & \textbf{Rigbyspace ontic structure}\\ \hline
Time & Causal index $k\in\mathbb{Z}_{\ge 0}$ and duration functional $T(\{s_k\})$\\
Position & Linear state $(n,d)\in S_L$\\
Momentum & Coupled evolution of $(n,d)$ under $\eta$ and $\psi$\\
Energy & Vacuum resolution frequency $\Omega_{\mathrm{vac}}$\\
Mass & Inertial basis $d$ and mass level $\rho(d)$\\
Wave-like regime & History-free duration condition $H=0$ in the duration functional\\
Particle-like regime & History-bearing condition $H>0$ in the duration functional\\
Symmetry & Closed phase cycle $\Phi_N$ and invariant cross-product relations\\
Field & Coupled state system on $S_L$ and $S_P$ under lawful transformations\\
Curvature / gravity & Non-vanishing drift $D(N)\ne 0$ in phase viscosity\\
Constant & Stable triadic barycentric cycle structure with unit-width lock\\
\end{longtable}
\end{center}

%=========================================================
\part{Glossary}
%=========================================================

\section{Glossary of Terms}

\begin{longtable}{>{\raggedright\arraybackslash}p{0.28\textwidth} >{\raggedright\arraybackslash}p{0.67\textwidth}}
\textbf{Term} & \textbf{Definition}\\ \hline
Accumulator $\lambda$ & Vacuum generator on $S_L$: $\lambda(n,d)=(n+d,d)$.\\
Arrow of Time & Matter admissibility law enforcing strict increase of structural depth, e.g.\ $\rho(Z')>\rho(Z)$.\\
Boost state $U$ & A linear state $(n_u,d_u)\in S_L$ used to define integer event transformations.\\
Causal index & The universal ordered label $k$ of lawful succession.\\
Constraint vacuum & A suspended linear state $(n,0)$ with $n\ne 0$.\\
Cross-determinant $\Delta_{\times}$ & Deviation between unreduced pairs: $\Delta_{\times}(a,b)=n_b d_a - n_a d_b$.\\
Doubling threshold $T_k$ & Integer-doubling sequence: $T_0=1$, $T_{k+1}=2T_k$.\\
Drift $D(N)$ & Phase viscosity drift: $D(N)=\sum_{k=0}^{T-1}\sigma(\tau_k)$.\\
Duration functional $T(\{s_k\})$ & Physical duration accumulated by a trajectory: $T=\sum \mathbf{1}_{H(s_k)>0}$.\\
Event $E$ & Ordered pair of linear states $(S_t,S_x)$.\\
Extended linear space $S_L^{\mathrm{ext}}$ & $S_L$ together with suspended states $(n,0)$.\\
Interval $I(E)$ & Integer interval: $I(E)=n_t^2 d_x^2 - n_x^2 d_t^2$.\\
Linear state space $S_L$ & Unreduced integer pairs $(n,d)$ with $d\ne 0$.\\
Mass level $m_L$ & Structural mass classification: $m_L=\rho(d)$.\\
Matter state space $S_P$ & Integer triples $(X,Y,Z)$ with $X,Y,Z\ne 0$.\\
Non-existence & Ontological exclusion of real numbers, limits, analytic functions, equivalence classes, and continuum structures.\\
Numerical ZERO & Linear state $(0,d)$ with $d>0$.\\
Phase space $\Phi_N$ & Directed cycle graph on vertices $\{0,\dots,N-1\}$.\\
Rank $\rho$ & Structural depth defined by doubling thresholds: $T_{\rho(m)}\le m<T_{\rho(m)+1}$.\\
Reciprocal $\psi$ & Coupled transformation: $\psi((a,b),(c,d))=((d,a),(b,c))$.\\
Resolution law & ZERO Logic rule: $\psi((n,0),(c,d))=((0,n),(d,c))$.\\
Separation principle & Ontic objects determine law; descriptive labels do not.\\
Successor-with-reset & Phase transformation $\mathrm{succ}_N$ advancing $s$ and resetting at $N-1$ to $0$.\\
Triadic convergence & Stabilization into the barycentric cycle \{Emission, Memory, Return\} with unit-width lock.\\
Vacuum resolution frequency $\Omega_{\mathrm{vac}}$ & Smallest $K$ for which rank increase meets a tension threshold along a lawful vacuum trajectory.\\
\end{longtable}

%=========================================================
\part{Conclusion}
%=========================================================

\section{Conclusion}
Rigbyspace Dynamics is a complete physical foundation expressed solely in integer ontology.
Cyclicity is a graph property, scale is rank, deviation is cross-determinant, stabilization is triadic lock,
vacuum evolution is linear, matter evolution is multiplicative and enforces monotone structural depth,
ZERO Logic resolves suspension without null collapse, interval structure is preserved through integer cross-relations,
and gravity appears as non-vanishing phase viscosity drift. Constants are stable barycentric cycle structures.

\end{document}
