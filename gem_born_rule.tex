\documentclass[11pt,a4paper]{article}
\usepackage[utf8]{inputenc}
\usepackage{amsmath, amssymb, amsthm}
\usepackage{geometry}
\usepackage{tikz}
\usepackage{xcolor}
\usepackage{booktabs}
\usetikzlibrary{shapes.geometric, arrows, positioning, calc}

\geometry{margin=1in}

\title{The Discrete Born Rule: \\
Constructive Probability from Integer State Superposition}

\author{D. Veneziano \\
\small{Department of Discrete Foundations}}
\date{January 2026}

\begin{document}

\maketitle

\begin{abstract}
We derive the Rigbyspace equivalent of the Born Rule ($P \propto |\psi|^2$) utilizing only the strict integer ontology of the framework. We reject the continuum postulates of complex amplitudes and real-valued probabilities in favor of a constructive approach based on \textbf{Explicit Rational States} (ERPs) and \textbf{Integer Magnitude Logic}. We demonstrate that "Interference" arises naturally from the signed addition of integer numerators in a superposition, while the "Square Law" emerges from the system's definition of Constructive Magnitude ($m^2$). This establishes a deterministic, integer-only pathway to quantum statistics, where probability is identified with the normalized structural intensity of the vacuum field.
\end{abstract}

\section{Part I: The Algebra of Discrete Superposition}

\subsection{1.1 The Ontological Primitive: Signed Integer States}
In standard quantum mechanics, the state is a vector in a complex Hilbert space. In Rigbyspace, the state is an \textbf{Explicit Rational Pair} (ERP) $s = (n, d) \in \mathbb{Z} \times \mathbb{Z}_{\ge 0}$ (Master Reference, Def 12.2).
Crucially, the numerator $n$ carries \textbf{sign information}, which corresponds to the chiral orientation (phase) of the state.

\begin{itemize}
    \item \textbf{Positive Phase:} $s_+ = (1, 1)$.
    \item \textbf{Negative Phase:} $s_- = (-1, 1)$. (Often generated by a $\pi$-shift or chiral flip).
\end{itemize}
Unlike the denominator $d$ (Inertial Mass), which is strictly non-negative, the numerator $n$ (Magnitude History) tracks the orientation in the phase space.

\subsection{1.2 The Principle of Linear Accumulation}
When multiple histories (paths) converge at a locus (e.g., a detector), the local vacuum state is defined by the \textbf{Linear Superposition} of the arriving states. In Rigbyspace, this is governed by the \textbf{Rational Addition Rule} (consistent with Algorithm 7.1 and Case C dynamics):

\textbf{Definition 1.1 (Discrete Superposition).}
Let two history states $s_1 = (n_1, d_1)$ and $s_2 = (n_2, d_2)$ arrive at a vertex. The resulting local state $S_{local}$ is the unreduced rational sum:
\begin{equation}
    S_{local} = s_1 \oplus s_2 := (n_1 d_2 + n_2 d_1, d_1 d_2)
\end{equation}
If the denominators are identical (coherent source, $d_1=d_2=d$), this simplifies to:
\begin{equation}
    S_{local} = (n_1 + n_2, d)
\end{equation}
(Note: We retain the common denominator for coherent summation to preserve mass-level).

This addition rule allows for \textbf{Cancellation without Subtraction operators}:
\begin{equation}
    (1, d) \oplus (-1, d) = (0, d)
\end{equation}
The result is the \textbf{Numerical ZERO} state (Def 12.1), which physically corresponds to a non-interacting vacuum photon or node.

\section{Part II: The Derivation of Quadratic Probability}

\subsection{2.1 Constructive Magnitude ($m^2$)}
Why is probability quadratic ($|\psi|^2$)? In Rigbyspace, this is not a postulate but a definition of how magnitude is measured.
Recall \textbf{Definition 6.10 (Constructive Magnitude)} from the Master Reference:
\begin{quote}
"For an integer $m$, magnitude is defined structurally via the self-product $m^2$. ... The symbol $|x|$ implies $\sqrt{x^2}$ and is forbidden."
\end{quote}
The system does not "know" the linear magnitude $|n|$. It only computes the \textbf{Structural Intensity} $I = n^2$ via the Integer Height Algorithm (Alg 7.2).

\subsection{2.2 The Probability Definition}
The "Probability" of an event is defined as the normalized Structural Intensity of the local state.

\textbf{Definition 2.2 (The Discrete Born Rule).}
Let $\Omega = \{S_1, S_2, \dots, S_k\}$ be the set of local states at mutually exclusive detectors. Let $S_i = (N_i, D_i)$. The probability $P(i)$ is the Explicit Rational Pair:
\begin{equation}
    P(i) := \left( N_i^2, \sum_{j=1}^k N_j^2 \right)
\end{equation}
This definition satisfies all RS constraints:
\begin{enumerate}
    \item \textbf{Integer Only:} $N^2$ is an integer.
    \item \textbf{Non-Negative:} Squares are always non-negative ($(-n)^2 = n^2$). This ensures probabilities are valid counts.
    \item \textbf{Quadratic Scaling:} It scales as the square of the summed numerators (amplitudes).
\end{enumerate}

\subsection{2.3 Red Team Analysis: Why this works}
\textbf{Objection:} "You just redefined probability to match the answer."
\textbf{Response:} No. We used the pre-existing definition of Magnitude from the framework (Def 6.10). In a discrete system without square roots, the "size" of a signed integer $n$ is naturally $n^2$ (its position in the quadratic form). If conservation laws apply to the \textit{energy} of the wave (which goes as amplitude squared), then the counting statistics of detection must reflect that intensity.

\textbf{Objection:} "Does this conserve probability?"
\textbf{Response:} Yes. By definition, the sum of the numerators of the probability pairs is $\sum N_i^2$, which matches the denominator. The sum is the Unit State $(1, 1)$ (or equivalent unity). Unitarity is preserved structurally.

\section{Part III: Worked Example - The Mach-Zehnder Interferometer}

We trace a single particle $(1,1)$ through a split, phase shift, and recombination.

\subsection{3.1 Setup}
\textbf{Source:} $s_{in} = (1, 1)$. (Unit Intensity $1^2 = 1$).
\textbf{Splitter (BS1):} Generates two paths. To conserve integer mass, we map $(1,1) \to \{(1,1), (1,1)\}$ (Duplication of history potential).
\textbf{Path 1:} State $h_1 = (1, 1)$.
\textbf{Path 2:} State $h_2 = (1, 1)$.

\subsection{3.2 Phase Shift ($\pi$ Rotation)}
A "$\pi$ shift" corresponds to a sign inversion of the numerator (Chiral flip).
\textbf{Path 2 Operation:} $h_2 \to h_2' = (-1, 1)$.

\subsection{3.3 Recombination (BS2)}
The beamsplitter mixes the paths. We use the standard Hadamard-like mixing logic (integer version):
\begin{itemize}
    \item \textbf{Port A (Constructive):} Sum of inputs. $S_A = h_1 \oplus h_2'$.
    \item \textbf{Port B (Destructive):} Difference of inputs. $S_B = h_1 \oplus (-h_2')$.
\end{itemize}
(Note: "Difference" means adding the sign-flipped state).

\textbf{Port A Calculation:}
$$ S_A = (1, 1) \oplus (-1, 1) = (1 + (-1), 1) = (0, 1) $$
This is the \textbf{Numerical ZERO} state.
\textbf{Intensity:} $I_A = 0^2 = 0$.

\textbf{Port B Calculation:}
$$ S_B = (1, 1) \oplus (-(-1, 1)) = (1, 1) \oplus (1, 1) = (2, 1) $$
State is $(2, 1)$.
\textbf{Intensity:} $I_B = 2^2 = 4$.

\subsection{3.4 Probabilities}
Total Intensity: $\Sigma I = 0 + 4 = 4$.
$$ P(A) = (0, 4) \equiv 0 $$
$$ P(B) = (4, 4) \equiv 1 $$

\textbf{Result:} Perfect interference. The "Dark Port" (A) has zero probability not because the particle vanished, but because the local vacuum state $(0,1)$ has zero structural magnitude. The "Bright Port" (B) captures all the intensity.

\subsection{3.5 Visualization}

%\begin{center}
%\begin{tikzpicture}[
%    node distance=1.5cm,
%    state/.style={rectangle, draw, rounded corners, minimum width=2cm, minimum height=0.8cm, align=center, fill=blue!10},
%    op/.style={diamond, draw, aspect=2, fill=yellow!10, font=\small},
%    arrow/.style={->, thick, >=stealth}
%]
%
%% Nodes
%\node[state] (source) {Source\\$s=(1,1)$};
%\node[op, right=of source] (split) {Split};
%
%\node[state, above right=of split] (path1) {Path 1\\$(1,1)$};
%\node[state, below right=of split] (path2) {Path 2\\$(1,1)$};
%
%\node[op, right=of path2] (shift) {$\pi$ Shift\\$n \to -n$};
%\node[state, right=of shift] (path2p) {Path 2'\\$(-1,1)$};
%
%\node[op, right=of path1, yshift=-1.5cm, xshift=2cm] (mix) {Recombine\\$\oplus$};
%
%\node[state, above right=of mix, fill=red!10] (portA) {Port A (Sum)\\$(0,1)$\\$I=0$};
%\node[state, below right=of mix, fill=green!10] (portB) {Port B (Diff)\\$(2,1)$\\$I=4$};
%
%% Arrows
%\draw[arrow] (source) -- (split);
%\draw[arrow] (split) |- (path1);
%\draw[arrow] (split) |- (path2);
%\draw[arrow] (path2) -- (shift);
%\draw[arrow] (shift) -- (path2p);
%
%\draw[arrow] (path1) -| node[near start, above] {$(1,1)$} (mix);
%\draw[arrow] (path2p) -| node[near start, below] {$(-1,1)$} (mix);
%
%\draw[arrow] (mix) -- (portA);
%\draw[arrow] (mix) -- (portB);
%
%% Annotations
%\node[right=0.5cm of portA] {Dark ($0^2$)};
%\node[right=0.5cm of portB] {Bright ($2^2$)};
%
%\end{tikzpicture}
%\end{center}

\end{document}